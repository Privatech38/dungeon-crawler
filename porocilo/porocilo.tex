
\documentclass[a4paper]{article}

\usepackage[utf8]{inputenc}
\usepackage{erk}
\usepackage{times}
\usepackage{graphicx}
\usepackage[top=22.5mm, bottom=22.5mm, left=22.5mm, right=22.5mm]{geometry}

\usepackage[slovene,english]{babel}
\usepackage{hyperref}
\usepackage{url}

\let\oldfootnotesize\footnotesize
\renewcommand*{\footnotesize}{\oldfootnotesize\scriptsize}

\begin{document}

\title{Poročilo seminarske naloge pri predmetu Računalniška grafika}
\author{Matic Grobolšek\and Žiga Slatnar Štagar}
\maketitle

\selectlanguage{slovene}

\begin{abstract}{Abstract}
Osnovna ideja igre je preprosta: igralec prične v prazni sobi in se 
mora prebiti skozi več drugih sob, kjer so lahko sovražniki, katere 
mora premagati, če želi napredovati.
Igra spada v žanr tako imenovanih ``dungeon crawler'' iger.
Z realizacijo igre smo se odločili uporabiti
programski jezik TypeScript in knjižnico WebGPU. Za izdelavo modelov,
tekstur in animacij smo uporabili Blender.
Igra je bila razvita za
spletni brskalnik.
\end{abstract}

\section{Pregled igre}\label{sec:pregled-igre}
Igra spada pod žanr ``dungeon crawler'' iger, ki pa spadajo pod žanr RPG oz.\ iger igranja vlog.
Igralec mora v igri napredovati skozi labirint, v tem primeru naključno ustvarjen iz sob.
V sobah se lahko nahajajo sovražniki, ki jih mora igralec premagati, da lahko napreduje.\\
Zahtevnost igre je statična in se je ne da prilagajati.
Igralec mora imeti dovolj sposobnosti, da se zna taktično premikati in pravilno uporabljati svoje orožje.\\
Ideja zgodbe je zelo preprosta, igralec se znajde v ``ječi'' in mora najti izhod iz nje.
To stori tako, da obišče vse sobe.

\subsection{Opis sveta}\label{subsec:opis-sveta}
Svet igre je sestavljen iz več naključno ustvarjenih sob, ki so med seboj povezane.
Stil sveta je srednjeveški, zato so tudi sovražniki in igralec srednjeveški liki.
Kot stil bi lahko opisali tudi tako imenovani ``low poly'' oz.\ stil, kjer so poligoni očitni.
Igralec, kot tudi sovražniki, se premikajo po dveh dimenzijah.

\subsubsection{Pregled}
Svet je naključno sestavljen ob začetku igre in sicer s tem, da uporablja le en model za zid, steber in vrata.
Vtis naključnosti je povečan z naključnim obračanjem teh modelov.

\subsubsection{Ozadje}
Ozadje je temno, namreč smo le v ječi in zato zunanje svetlobe ni ali pa je minimalna.

\subsubsection{Ključne lokacije}
Posebnih lokacij ni, razen soba, v kateri igralec prične, kjer se sovražniki ne morejo pojaviti.

\subsubsection{Velikost}
Vsak zid je dolg tri metre.\ Skupna površina sob ima največjo vrednost, ki hkrati med ustvarjanjem svet
definira število sob, saj se program za ustvarjanje ustavi, ko doseže to vrednost.

\subsubsection{Objekti}
Vsi objekti v igri so ročno ustvarjeni v programih
Blender\footnote{\url{https://www.blender.org/}} in Paint.NET\footnote{\url{https://www.getpaint.net/}}.

\subsubsection{Čas}
Čas v igri je statičen oziroma ni definiran, saj igra nima nobenega elementa, ki bi bil odvisen od internega časa.
Če pa bi primerjali z resničnim svetom, bi vzeli ta čas.

\subsection{Igralni pogon in uporabljene tehnologije}\label{subsec:igralni-pogon-in-uporabljene-tehnologije}
Igralni pogon je napisan v programskem jeziku TypeScript\footnote{\url{https://www.typescriptlang.org/}}.
Za izrisovanje uporablja knjižnico WebGPU\footnote{\url{https://www.w3.org/TR/webgpu/}}.
Kot povezava med najinim gonilnikom in WebGPU je uporabljen repozitorij webgpu-examples\footnote{\url{https://github.com/UL-FRI-LGM/webgpu-examples}} od UL FRI LGM.
Za izdelavo modelov, UV koordinat in animacij smo uporabili Blender\footnote[1].
Za izdelavo tekstur pa Paint.NET\footnote[2].

\subsection{Pogled}\label{subsec:pogled}
Za pogled je uporabljena perspektivna kamera, ki je postavljena nad igralcem in obrnjena proti njemu.
Kamera se premika skupaj z igralcem, vendar se ne obrača z njim.

\section{Osebek}\label{sec:osebek}
Igralčev osebek predstavlja srednjeveškega viteza, sicer brez rok in nog, zaradi stila igre (tudi zaradi lažjega animiranja).
Igralec osebka premika z uporabo tipk W, A, S in D. Osebek se obrne v smeri premikanja.
Igralec pa nima nadzora nad sovražniki, v našem primeru so to okostnjaki, prav tako brez rok in nog.

\section{Uporabniški vmesnik}\label{sec:uporabniski-vmesnik}
Samega uporabniškega vmesnika ni.

\section{Glasba in zvok}\label{sec:glasba-in-zvok}
Glasba in zvok nista prisotna.

\section{Gameplay}\label{sec:gameplay}
Igra se prične tako, da se igralec nahaja v prvi sobi, kjer ni sovražnikov.
Igralec mora v vsaki sobi ubiti vse sovražnike.
Ko to stori, se igra zaključi.

\section{Zaključki in možne nadgradnje}\label{sec:zakljucki-in-mozne-nadgradnje}
Pri izdelavi igre sva se največ naučila o WebGPU knjižnici\footnote[4].
Ker je WebGPU relativno nov pojem, sva za pomoč uporabljala repozitorij webgpu-examples\footnote[5] in repozitorij vaj\footnote{\url{https://github.com/UL-FRI-LGM/vaje-rgti}}.
Poleg tega sva nadgradila znanje TypeScript\footnote[3] jezika.
Prvič sva se srečala z glTF\footnote{\url{https://www.khronos.org/gltf/}} formatom datotek za shranjevanje modelov in njihovih podatkov.
\\
Predmet sam nama je dal dovolj znanja za razumevanje senčilnikov in kako jih pisati.
\\
Predviden scenarij je bilo dosti več nivojev, v smislu okolice npr. travnik, gozd, nebo itd.
Prav tako bi bilo več sovražnikov, orožij in celo uroki.
Dodati sva želela tudi gospodarja sovražnikov, ki bi bil zadnji sovražnik igre oz. nivoja.

\small
\bibliographystyle{plain}
\bibliography{references}

\end{document}