\documentclass[twocolumn]{article}

\title{Poročilo seminarske naloge pri predmetu Računalniška grafika}
\author{Bor Furlan \and Matic Grobolšek\and Žiga Slatnar Štagar}
\date{\today}

\begin{document}

\maketitle

\begin{abstract}
Osnovna ideja igre je preprosta: igralec prične v prazni sobi in se 
mora prebiti skozi več drugih sob, kjer so lahko sovražniki, katere 
mora premagati, če želi napredovati.
Igra spada v žanr tako imenovanih ``dungeon crawler'' iger.
Z realizacijo igre smo se odločili uporabiti
programski jezik TypeScript in knjižnico WebGPU. Za izdelavo modelov,
tekstur in animacij smo uporabili Blender.
Igra je bila razvita za
spletni brskalnik.
\end{abstract}

\section{Pregled igre}
Igra spada pod žanr ``dungeon crawler'' iger, ki pa spadajo pod žanr RPG oz. iger igranja vlog.
Igralec mora v igri napredovati skozi labirint, v tem primeru naključno ustvarjen iz sob.
V sobah se lahko nahajajo sovražniki, ki jih mora igralec premagati, da lahko napreduje.\\
Zahtevnost igre je statična in se je ne da prilagajati.
Igralec mora imeti dovolj sposobnosti, da se zna taktično premikati in pravilno uporabljati svoje orožje.\\
Ideja zgodbe je zelo preprosta, igralec se znajde v ``ječi'' in mora najti izhod iz nje.
To stori tako, da obišče vse sobe.

\subsection{Opis sveta}
\subsubsection{Pregled}
\subsubsection{Ozadje}
\subsubsection{Ključne lokacije}
\subsubsection{Velikost}
\subsubsection{Objekti}
\subsubsection{Čas}

\subsection{Igralni pogon in uporabljene tehnologije}

\subsection{Pogled}

\section{Osebek}

\section{Uporabniški vmesnik}

\section{Glasba in zvok}

\section{Gameplay}

\section{Zaključki in možne nadgradnje}

\end{document}